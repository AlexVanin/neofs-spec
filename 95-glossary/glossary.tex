\newacronym{dApp}{dApp}{Decentralized Application}
\newacronym{api}{API}{Application Programming Interface}
\newacronym{acl}{ACL}{Access Control List}
\newacronym{cli}{CLI}{Command Line Interface}

\newglossaryentry{mainnet}{
  name={N3 Main Net},
  description={
    N3 Main Network Blockchain.
    See \href{https://docs.neo.org/v3/docs/en-us/network/testnet.html}{Neo Documentation}
  }
}

\newglossaryentry{testnet}{
  name={N3 Test Net},
  description={
    N3 Main Network Blockchain.
    See \href{https://docs.neo.org/v3/docs/en-us/network/testnet.html}{Neo Documentation.}
  }
}

\newglossaryentry{NEO}{
  name={NEO Token},
  description={
    Token representing a share of ownership in the NEO blockchain
  }
}

\newglossaryentry{NeoVM}{
  name={Neo Virtual Machine},
  description={
    NeoVM is a lightweight virtual machine for executing Neo smart contracts. As the core component of Neo, NeoVM has Turing completeness and high consistency, which can implement arbitrary execution logic and ensure consistent execution results of any node in distributed network, providing strong support for decentralized applications.
    See \href{https://docs.neo.org/docs/en-us/basic/technology/neovm.html}{Neo Documentation.}
  }
}


\newglossaryentry{GAS}{
  name={GAS Utility Token},
  description={
    Neo Blockchain's utility token
  }
}

\newglossaryentry{validator}{
  name={validator node},
  description={
    In the NEO network, NEO holders can enroll themselves to be validators (consensus node candidates), and then be voted as consensus nodes. The voting status of validators and number of consensus nodes are stored in blockchain.
  }
}

\newglossaryentry{NeoFS}{
  name=NeoFS,
  description={
    \textbf{N}eoFS \textbf{E}xaggerative \textbf{O}bject \textbf{F}ile \textbf{S}torage. Also known as \textbf{Neo} \textbf{F}ile \textbf{S}torage
  }
}

\newglossaryentry{Node}{
  name=NeoFS Node,
  description={
  Computer system with at least following properties:
  \begin{itemize}
    \item Running relevant version of NeoFS software
    \item Has a NeoFS Network-wide unique identifier and a key pair
    \item Has good enough connectivity with other nodes
    \item Serves requests using NeoFS API protocol
  \end{itemize}
  }
}

\newglossaryentry{EigenTrust}{
  name=EigenTrust,
  description={
    EigenTrust algorithm is a reputation management \href{http://ilpubs.stanford.edu:8090/562/1/2002-56.pdf}{algorithm} for peer-to-peer networks.
  }
}

\newglossaryentry{HRW}{
  name=HRW,
  description={
    HRW stands for \href{https://en.wikipedia.org/wiki/Rendezvous_hashing}{Rendezvous hashing}. It helps to achieve 3 goals:
    \begin{enumerate}
      \item Select nodes uniformly from the whole netmap. This means that every node has a chance to be included in container nodes set.
      \item Select nodes deterministically. Identical (netmap, storage policy) pair results in the same placement set on every storage node.
      \item Prioritize nodes providing better conditions.
    \end{enumerate}
    Nodes having more space, better price or better rating are to be selected with higher probability. Specific weighting algorithm is defined for NeoFS network as a whole and is beyond scope of this document. See \href{https://github.com/nspcc-dev/hrw}{NeoFS HRW implementation} for details.
  }
}

\newglossaryentry{LocalTrust}{
  name=Local Trust,
  description={
    trust of one node to another, calculated using \textit{only} statistical information of their peer-to-peer network interactions. The Subject and Object of such a trust are peer-to-peer nodes.
  }
}

\newglossaryentry{GlobalTrust}{
  name=Global Trust,
  description={
    the result of the \Gls{EigenTrust} algorithm is the trust in the network participant, which has been obtained regarding \textit{all} \Gls{LocalTrust}s of \textit{all} nodes.
  }
}
