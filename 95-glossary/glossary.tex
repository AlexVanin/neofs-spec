\newacronym{dApp}{dApp}{Decentralized Application}
\newacronym{api}{API}{Application Programming Interface}
\newacronym{acl}{ACL}{Access Control List}
\newacronym{cli}{CLI}{Command Line Interface}

\newglossaryentry{mainnet}{
  name={Neo 3.0 Main Net},
  description={
    Neo 3.0 Main Network Blockchain.
    See \href{https://docs.neo.org/v3/docs/en-us/network/testnet.html}{Neo Documentation}
  }
}

\newglossaryentry{testnet}{
  name={Neo 3.0 Test Net},
  description={
    Neo 3.0 Main Network Blockchain.
    See \href{https://docs.neo.org/v3/docs/en-us/network/testnet.html}{Neo Documentation.}
  }
}

\newglossaryentry{NEO}{
  name={NEO Token},
  description={
    Token representing a share of ownership in the NEO blockchain
  }
}

\newglossaryentry{NeoVM}{
  name={Neo Virtual Machine},
  description={
    NeoVM is a lightweight virtual machine for executing Neo smart contracts. As the core component of Neo, NeoVM has Turing completeness and high consistency, which can implement arbitrary execution logic and ensure consistent execution results of any node in distributed network, providing strong support for decentralized applications.
    See \href{https://docs.neo.org/docs/en-us/basic/technology/neovm.html}{Neo Documentation.}
  }
}


\newglossaryentry{GAS}{
  name={GAS Utility Token},
  description={
    Neo Blockchain's utility token
  }
}

\newglossaryentry{validator}{
  name={validator node},
  description={
    In the NEO network, NEO holders can enroll themselves to be validators (consensus node candidates), and then be voted as consensus nodes. The voting status of validators and number of consensus nodes are stored in blockchain.
  }
}

\newglossaryentry{NeoFS}{
  name=NeoFS,
  description={
    \textbf{N}eoFS \textbf{E}xaggerative \textbf{O}bject \textbf{F}ile \textbf{S}torage. Also known as \textbf{Neo} \textbf{F}ile \textbf{S}torage
  }
}

\newglossaryentry{Node}{
  name=NeoFS Node,
  description={
  Computer system with at least following properties:
  \begin{itemize}
    \item Running relevant version of NeoFS software
    \item Has a NeoFS Network-wide unique identifier and a key pair
    \item Has good enough connectivity with other nodes
    \item Serves requests using NeoFS API protocol
  \end{itemize}
  }
}
